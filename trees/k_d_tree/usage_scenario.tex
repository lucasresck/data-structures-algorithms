\section{Cenário de uso}

    O cenário de uso da nossa implementação consiste em:
    \begin{itemize}
        \item dada a localização do fiscal, indicar o hospital mais próximo;
        \item dado o hospital, indicar os $x$ pacientes mais similares parecidos ao informado.
    \end{itemize}
    Para isso, precisamos criar um \textit{dataset} com essas informações.

    Toda a preparação do cenário de uso, incluindo a geração do \textit{dataset}, podem ser visualizadas \href{https://nbviewer.jupyter.org/github/lucasresck/data-structures-algorithms/blob/main/notebooks/k_d_tree.ipynb}{neste \textit{Jupyter notebook}}. O \textit{dataset}, por sua vez, pode ser consultado \href{https://github.com/lucasresck/data-structures-algorithms/blob/main/data/hospitals.json}{neste arquivo JSON}.

    \subsection{\textit{Dataset}}

        Nosso \textit{dataset} contém 10 hospitais, com 7 pacientes cada. Por facilidade, cada hospital possui uma localização aleatória dentro de um quadrado unitário, e cada paciente possui características aleatórias dentro dos limites informados no enunciado.
        Veja, por exemplo, o início do arquivo JSON gerado:
        \begin{alltt}
\{
    "hospitais": [
        \{
            "localização": [
                0.62,
                0.61
            ],
            "pacientes": [
                \{
                    "peso": 91,
                    "idade": 29,
                    "saturação": 90
                \},
                \{
                    "peso": 111,
                    "idade": 75,
                    "saturação": 100
                \}, ...
        \end{alltt}
        Cada hospital é um objeto que contém sua localização e uma lista de seus pacientes; o \textit{dataset}, portanto, é uma lista desses hospitais.

    \subsection{Hospital mais próximo}

        Suponhamos que o fiscal esteja na posição $(0.5, 0.5)$ do quadrado unitário. A \autoref{fig:points} mostra a localização dos hospitais e do fiscal.
        \begin{figure}[H]
            \centering
            \includegraphics[width=0.6\textwidth]{points.pdf}
            \caption{Localização dos hospitais (em azul) e do fiscal (em laranja).}
            \label{fig:points}
        \end{figure}
        Depois de montarmos a árvore k-d utilizando a localização dos hospitais como nó, realizamos uma busca k-NN com $k$ igual a 1 para encontrarmos o hospital mais próximo. A saída do algoritmo é
        \begin{alltt}
[\{'node': [0.56, 0.53],
  'metadata': \{'localização': [0.56, 0.53],
   'pacientes': [\{'peso': 98, 'idade': 50, 'saturação': 98\},
    \{'peso': 129, 'idade': 81, 'saturação': 98\},
    \{'peso': 59, 'idade': 85, 'saturação': 99\},
    \{'peso': 72, 'idade': 54, 'saturação': 86\},
    \{'peso': 121, 'idade': 15, 'saturação': 90\},
    \{'peso': 96, 'idade': 64, 'saturação': 89\},
    \{'peso': 46, 'idade': 30, 'saturação': 95\}]\}\}]
        \end{alltt}
        Encontramos o hospital localizado no ponto $(0.56, 0.53)$, realmente o ponto mais próximo na \autoref{fig:points}. Além disso, pelo fato de que nossa implementação permite o armazenamento de metadados, temos também todas as informações desse hospital, incluindo a lista de pacientes e suas características.

    \subsection{Pacientes mais parecidos}
