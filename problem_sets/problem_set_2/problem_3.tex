\noindent \textbf{Exercício 3} (2.5 pontos) Considerando o problema da mochila apresentado
previamente. Vamos supor que ao invés de somente um valor de prêmio
$v_i$ para cada item da mochila, existam $m$ valores de prêmio, e com isso
teremos múltiplos objetivos. Desenhe uma busca $A^\ast$ ou branch-and-bound
para esse problema.

\bigskip

\noindent \textbf{Solução:}
Inicialmente, vamos tratar da mochila uniobjetivo, para, posteriormente, generalizar o resultado para a mochila multiobjetivo. Para o problema da mochila original (uniobjetivo), podemos encontrar uma solução relaxada utilizando o algoritmo de \href{https://en.wikipedia.org/wiki/Continuous_knapsack_problem}{Dantzig}: um algoritmo guloso que enche a mochila com frações dos itens mais valiosos, ordenados pelo seu custo-benefício (valor dividido por peso), até que a mochila esteja cheia. Caso a solução resultante seja inteira, está tudo certo, e essa é a solução final.

Caso a solução para os itens não seja inteira (mais especificamente, as quantidades dos itens estejam entre $0$ e $w_i$), precisaremos ramificar. Ramificamos no item $i$ que possui o valor fracionário mais próximo de $0.5w_i$, e iniciamos na ramificação $0$ ou $w_i$, dependendo de o valor relaxado estar mais próximo de  $0$ ou $w_i$. Isso descreve completamente a sequência de $branches$.

Toda vez que calculamos um novo nó após uma ramificação, também encontramos o valor da mochila uniobjetivo dessa configuração relaxada. Isso significa que, naquela subárvore, todas as soluções, inteiras ou não, têm, no máximo, esse mesmo valor. Podemos armazenar uma variável chamada \textit{limitante inferior} de tal forma que, quando nos deparamos com uma subárvore que tem solução ótima contínua menor do que o limitante inferior, não chegamos nem a explorar as ramificações dessa árvore. Em outras palavras, só buscamos soluções melhores do que já temos.

O algoritmo descrito acima possui forte analogia com o branch-and-bound para problemas de programação linear, pelo menos no caso uni-objetivo.

Podemos generalizar esse algoritmo para o caso multiobjetivo da mochila. A princípio, o algoritmo permanece o mesmo para o \textit{branch} nas variáveis não inteiras. Porém, agora, vamos tomar outra estratégia para o \textit{bound}: vamos manter as soluções multiobjetivo armazenadas em uma lista chamada \textit{soluções não dominadas}. Quando formos decidir explorar um nó, calculamos, para cada um dos $m$ valores dos itens, a solução ótima contínua, e reproduzimos essa configuração para os outros valores. Ou seja, para cada $i$, teremos uma solução multiobjetivo, porém que maximizou $i$ apenas, e serão $m$ soluções desse tipo. Na nossa lista \textit{soluções não dominadas}, buscamos por uma solução que domine \textbf{todas} as nossas $m$ soluções encontradas no nó. É importante uma mesma solução dominar todas, porque aí garantimos que essa solução domina qualquer solução intermediária, que não maximiza nenhum $0 \le i \le m$ particularmente. Só não vamos explorar a subárvore desse nó caso encontrarmos essa solução dominante. Se as $m$ soluções da mochila contínua em um nó são dominadas por alguma outra solução já armazenada, essa solução armazenada tem cada uma das suas $m$ coordenadas da solução multiobjetivo melhor do que cada uma das soluções uniobjetivo contínuas do nó. Ou seja, essa solução armazenada necessariamente domina qualquer solução multiobjetivo inteira da subárvore. Isso nos faz decidir, ou não, se exploramos a subárvore.

Toda vez que chegamos em uma folha (\textit{leaf}), verificamos se essa solução é não dominada. Caso positivo, inserimos ela em nossa lista. Isso nos garante que, ao final do algoritmo, teremos apenas as \textbf{soluções não dominadas}, pois todas as dominadas foram descartadas.
