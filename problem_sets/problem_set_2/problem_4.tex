\noindent \textbf{Exercício 4} (3.0 pontos) Suponha que temos $n$ tarefas onde cada uma toma
um tempo $t_i$ para processar em $m$ máquinas idênticas nas quais desejamos
dizer qual a sequência de tarefas em cada máquina. Uma mesma tarefa
não pode ser divida entre máquinas. Para um dado planejamento,
$A_j$ é o conjunto de tarefas colocados na máquina $j$.
Seja $L_j = \sum_{i \in A_j} t_i$ o tempo da
máquina $j$. O tempo de execução dessas tarefas será o maior $L_j$ dentre as
máquinas.
Nós consideramos um algoritmo guloso para esse problema que ordena
as tarefas de tal forma que $t_1 \ge t_2 \ge \cdots \ge t_n$ e iterativamente aloca o
próximo serviço na maquina com a menor carga.
Demonstre quão boa é a aproximação deste algoritmo (considerando OPT
é a solução ótima, esperamos algo menor que 2OPT).

\bigskip

\noindent \textbf{Solução:}
Chamemos de $\text{OPT}^\ast$ a solução ótima para esse problema, porém relaxando a condição de que as tarefas não podem ser divididas entre máquinas. Dessa forma, é claro que $\text{OPT}^\ast = \frac{1}{m} \sum_{i = 1}^n t_i$, quando todas as máquinas possuem tempo exatamente igual. (Suponha, por absurdo, que esse não seja o caso. Então alguma máquina terá tempo menor que $\text{OPT}^\ast$, porém alguma outra máquina terá necessariamente tempo maior do que $\text{OPT}^\ast$. Ora, é possível reduzir o tempo de uma das máquinas, diminuindo o tempo de execução total. Absurdo.) É interessante pensar nesse problema como barras de altura $t_i$, que devem ser empilhadas em $m$ posições diferentes, de modo a minimizar a altura da maior das $m$ torres. Essa é a analogia que seguiremos daqui para frente.

Voltemos ao problema original. Afirmação: não existe barra que tenha sua base a partir da altura $\text{OPT}^\ast$, isto é, toda barra tem sua base iniciando em uma altura menor do que $\text{OPT}^\ast$. Ora, não é difícil ver que, caso existisse essa barra, quando ela foi colocada é necessariamente verdade que existia pelo menos uma torre com altura menor do que $\text{OPT}^\ast$, e essa barra deveria ter sido empilhada nessa torre. Para justificarmos essa segunda afirmação, podemos argumentar que, caso não houvesse torre com altura menor do que $\text{OPT}^\ast$, todas as torres teriam altura a partir de $\text{OPT}^\ast$, de modo que a média das alturas das torres fosse maior do que ou igual a $\text{OPT}^\ast$, o que é um absurdo, claramente, pois contradiz a própria definição de $\text{OPT}^\ast$.

Ora, mostramos, basicamente, que $G \le \text{OPT}^\ast + \max t_i$. Daí segue que:
\begin{align*}
    G &\le \text{OPT}^\ast + \max t_i \\
    &\le \text{OPT} + \max t_i \\
    &\le \text{OPT} + \text{OPT} \\
    &\le 2\text{OPT},
\end{align*}
sendo a penúltima desigualdade justificada com o fato de que, se não é possível distribuir frações das barras, então necessariamente $t_i \le \text{OPT}$, para todo $i$.
