\noindent \textbf{Exercício 4} (3.0 pontos) Suponha que temos $n$ tarefas onde cada uma toma
um tempo $t_i$ para processar em $m$ máquinas idênticas nas quais desejamos
dizer qual a sequência de tarefas em cada máquina. Uma mesma tarefa
não pode ser divida entre máquinas. Para um dado planejamento,
$A_j$ é o conjunto de tarefas colocados na máquina $j$.
Seja $L_j = \sum_{i \in A_j} t_i$ o tempo da
máquina $j$. O tempo de execução dessas tarefas será o maior $L_j$ dentre as
máquinas.
Nós consideramos um algoritmo guloso para esse problema que ordena
as tarefas de tal forma que $t_1 \ge t_2 \ge \cdots \ge t_n$ e iterativamente aloca o
próximo serviço na maquina com a menor carga.
Demonstre quão boa é a aproximação deste algoritmo (considerando OPT
é a solução ótima, esperamos algo menor que 2OPT).

\bigskip

\noindent \textbf{Solução:}
