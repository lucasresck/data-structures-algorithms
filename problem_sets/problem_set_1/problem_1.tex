\noindent \textbf{Exercício 1} (2.5 pontos) Escolha 20 cidades brasileiras, encontre as distâncias
entre elas de forma euclideana e calculando sua distância usando estradas.
Mostre a execução do algoritmo A* para essas cidades (pode ser feito
tanto manualmente quando implementando o algoritmo e mostrando o
log de execução).

\bigskip

\noindent \textbf{Solução:} O algoritmo foi implementado para encontrar o menor caminho entre duas capitais brasileiras. O código está disponível neste \href{https://github.com/lucasresck/data-structures-algorithms/blob/main/notebooks/city_distances.ipynb}{Jupyter notebook}. Por exemplo, calculando o melhor caminho entre Porto Alegre e Rio Branco, encontramos o seguinte log da implementação:

\begin{verbatim}
Porto Alegre    0

Porto Alegre    Florianópolis    476

Porto Alegre    Curitiba    711

Porto Alegre    Campo Grande    1518

Porto Alegre    Florianópolis    Curitiba    776

Porto Alegre    Curitiba    Campo Grande    1702

Porto Alegre    Florianópolis    Curitiba    Campo Grande    1767

Porto Alegre    Florianópolis    Campo Grande    1774

Porto Alegre    Cuiabá    2206

Porto Alegre    Campo Grande    Cuiabá    2212

Porto Alegre    Florianópolis    Porto Alegre    952

Porto Alegre    Curitiba    Cuiabá    2390

Porto Alegre    Curitiba    Campo Grande    Cuiabá    2396

Porto Alegre    São Paulo    1108

Porto Alegre    Curitiba    Florianópolis    1011

Porto Alegre    Curitiba    São Paulo    1118

Porto Alegre    Florianópolis    Curitiba    Cuiabá    2455

Porto Alegre    Florianópolis    Curitiba    Campo Grande    Cuiabá    2461

Porto Alegre    Florianópolis    Cuiabá    2462

Porto Alegre    Florianópolis    Campo Grande    Cuiabá    2468

Porto Alegre    Florianópolis    Curitiba    Florianópolis    1076

Porto Alegre    Florianópolis    São Paulo    1181

Porto Alegre    Florianópolis    Curitiba    São Paulo    1184

Porto Alegre    Curitiba    Florianópolis    Curitiba    1311

Porto Alegre    São Paulo    Campo Grande    2122

Porto Alegre    Curitiba    São Paulo    Campo Grande    2132

Porto Alegre    Florianópolis    Curitiba    Florianópolis    Curitiba    1376

Porto Alegre    Goiânia    1847

Porto Alegre    Florianópolis    São Paulo    Campo Grande    2195

Porto Alegre    Florianópolis    Curitiba    São Paulo    Campo Grande    2198

Porto Alegre    Curitiba    Goiânia    1897

Porto Alegre    Florianópolis    Curitiba    Goiânia    1962

Porto Alegre    Florianópolis    Goiânia    1969

Porto Alegre    Porto Velho    3662

Porto Alegre    Cuiabá    Porto Velho    3662

Porto Alegre    São Paulo    Curitiba    1516

Porto Alegre    Campo Grande    Porto Velho    3668

Porto Alegre    Campo Grande    Cuiabá    Porto Velho    3668

Porto Alegre    Curitiba    São Paulo    Curitiba    1526

Porto Alegre    Curitiba    Florianópolis    Curitiba    Campo Grande    2302

Porto Alegre    São Paulo    Cuiabá    2722

Porto Alegre    Curitiba    Florianópolis    Campo Grande    2309

Porto Alegre    Curitiba    São Paulo    Cuiabá    2732

Porto Alegre    São Paulo    Goiânia    2034

Porto Alegre    Curitiba    São Paulo    Goiânia    2044

Porto Alegre    Florianópolis    São Paulo    Curitiba    1589

Porto Alegre    Florianópolis    Curitiba    São Paulo    Curitiba    1592

Porto Alegre    Rio Branco    4196
\end{verbatim}

Esses foram os nós verificados pelo algoritmo na fronteira da árvore de busca. Para cada linha, a lista mostra um histórico de caminho da origem até o nó verificado, e o número é o custo da origem até esse nó pelo caminho escolhido.
Por exemplo, a última linha do log mostra que o algoritmo escolheu ir de Porto Alegre para Rio branco diretamente, com custo 4196, curiosamente. Isso significa que existe (ou deve existir) uma estrada que liga Porto Alegre a Rio Branco mas que não passa pelas outras capitais.

Um ponto interessante é que, intuitivamente, o caminho mais óbvio seria Porto Alegre, Campo Grande, Cuiabá, Porto Velho e Rio Branco. Inclusive, o algoritmo chega até Porto Velho por esse caminho e encontra um custo de 3668. Considerando que a distância por estradas entre Porto Velho e Rio Branco é de 544, então seguir por esse caminho custaria 4212.
Ou seja, nosso algoritmo encontrou, realmente, um caminho de menor custo.
