\noindent \textbf{Exercício 1} (2.5 pontos) Escolha 20 cidades brasileiras, encontre as distâncias
entre elas de forma euclideana e calculando sua distância usando estradas.
Mostre a execução do algoritmo A* para essas cidades (pode ser feito
tanto manualmente quando implementando o algoritmo e mostrando o
log de execução).

\smallskip

\noindent \textbf{Bônus} (0.5) Calcule o índice de desvio dessas cidades e adapte a heurística
para que ela seja não admissível. O ótimo deixa de ser atingido? Qual a
diferença?

\bigskip

\noindent \textbf{Solução:} O algoritmo foi implementado para encontrar o menor caminho entre duas capitais brasileiras. O código está disponível neste \href{https://github.com/lucasresck/data-structures-algorithms/blob/main/notebooks/city_distances.ipynb}{Jupyter notebook}. Por exemplo, calculando o melhor caminho entre São Paulo e Salvador, encontramos o seguinte log da implementação:

\begin{verbatim}
[''] São Paulo

['', 'São Paulo'] Belo Horizonte

['', 'São Paulo'] Rio de Janeiro

['', 'São Paulo'] Vitória

['', 'São Paulo', 'Rio de Janeiro'] Vitória

['', 'São Paulo', 'Rio de Janeiro'] Belo Horizonte

['', 'São Paulo', 'Belo Horizonte'] Vitória

['', 'São Paulo', 'Belo Horizonte'] Salvador
\end{verbatim}

Essas foram as tentativas do algoritmo. Ele passou pelos nós de SP, BH, RJ, Vitória e, finalmente, Salvador. A última linha mostra o caminho escolhido, passando por SP, BH e indo direto para Salvador.
