\documentclass{article}

% Language setting
% Replace `english' with e.g. `spanish' to change the document language
\usepackage[portuguese]{babel}

% Set page size and margins
% Replace `letterpaper' with`a4paper' for UK/EU standard size
\usepackage[a4paper,top=2cm,bottom=2cm,left=3cm,right=3cm,marginparwidth=1.75cm]{geometry}

% Useful packages
\usepackage{amsmath}
\usepackage{graphicx}
\usepackage[colorlinks=true, allcolors=blue]{hyperref}
\usepackage{float}

\graphicspath{{images/}}

\title{Lista de exercícios 3}
\author{Lucas Emanuel Resck Domingues}

\begin{document}

    \maketitle

    \noindent \textbf{Problema:} Suponha que temos $n$ tarefas onde cada uma toma um tempo $t_i$ para processar em $m$ máquinas idênticas nas quais desejamos dizer qual a sequência de tarefas em cada máquina. Uma mesma tarefa não pode ser divida entre máquinas. Para um dado planejamento, $A_j$ é o conjunto de tarefas colocados na máquina $j$. Seja $L_j = \sum_{i \in A_j} t_i$ o tempo da máquina $j$. O tempo de execução dessas tarefas será o maior $L_j$ dentre as máquinas.

    \bigskip

    \noindent \textbf{Exercício 1} (2.0 pontos) Proponha uma representação/estrutura de dados para o problema acima.

    \bigskip

    \noindent \textbf{Solução:}


    \bigskip

    \noindent \textbf{Exercício 2} (2.0 pontos) Proponha uma heurística construtiva para o problema acima.

    \bigskip

    \noindent \textbf{Exercício 3} (2.0 pontos) Proponha uma busca local para o problema acima.

    \bigskip

    \noindent \textbf{Exercício 4} (2.0 pontos) Como você estruturaria a lista restrita de candidatos da heurística construtiva?

    \bigskip

    \noindent \textbf{Exercício 5} (2.0 pontos) Como você estruturaria os movimentos tabu para a busca local acima?

    \bigskip

    \noindent \textbf{Exercício 6} (2.0 pontos) Proponha uma estratégia de cruzamento e mutação para esse problema. Lembre-se a representação pode atrapalhar ou te ajudar muito na hora de fazer esse processo.

\end{document}