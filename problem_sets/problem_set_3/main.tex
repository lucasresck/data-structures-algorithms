\documentclass{article}
\usepackage[utf8]{inputenc}

\title{Lista de exercícios 3}
\author{Marcos Medeiros Raimundo}
\date{July 2021}

\usepackage{natbib}
\usepackage{graphicx}

\begin{document}

\maketitle

Problema: Suponha que temos $n$ tarefas onde cada uma toma um tempo $t_i$ para processar em $m$ máquinas idênticas nas quais desejamos dizer qual a sequência de tarefas em cada máquina. Uma mesma tarefa não pode ser divida entre máquinas. Para um dado planejamento, $A_j$ é o conjunto de tarefas colocados na máquina $j$. Seja $L_j = \sum_{i \in A_j} t_i$ o tempo da máquina $j$. O tempo de execução dessas tarefas será o maior $L_j$ dentre as máquinas.

\begin{description}

\item[Exercício 1] (2.0 pontos) Proponha uma representação/estrutura de dados para o problema acima.

\item[Exercício 2] (2.0 pontos) Proponha uma heurística construtiva para o problema acima.

\item[Exercício 3] (2.0 pontos) Proponha uma busca local para o problema acima.

\item[Exercício 4] (2.0 pontos) Como você estruturaria a lista restrita de candidatos da heurística construtiva?

\item[Exercício 5] (2.0 pontos) Como você estruturaria os movimentos tabu para a busca local acima?

\item[Exercício 6] (2.0 pontos) Proponha uma estratégia de cruzamento e mutação para esse problema. Lembre-se a representação pode atrapalhar ou te ajudar muito na hora de fazer esse processo.
\end{description}
\end{document}