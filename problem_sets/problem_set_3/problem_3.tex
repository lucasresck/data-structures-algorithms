\noindent \textbf{Solução:}
Seja $r \in \{1, \cdots, n\}$ fixo.
Desaloca-se, aleatoriamente, $r$ tarefas de suas respectivas máquinas (isso equivale a, no vetor representação $\text{\fbox{$j_1$}} \text{\fbox{$j_2$}} \cdots \text{\fbox{$j_n$}}$, substituir $r$ das entradas por um valor vazio).
Dentre os possíveis planejamentos que contêm o planejamento dado pela representação atual, calcula-se aquele que tem o menor tempo de execução $\min L_j$. Se esse novo planejamento é de alguma forma melhor do que o planejamento da iteração anterior, atualiza-se a solução e repete-se esses passos; em caso negativo, finalizamos o procedimento com a solução anterior. veja o Pseudocódigo \autoref{alg:local_search}.

\begin{algorithm}
    \caption{Busca local para a alocação de máquinas.}
    \label{alg:local_search}
    \begin{algorithmic}
        \Require $1 \le r \le n$
        \Require Solução inicial $s_0 = \text{\fbox{$j_1$}} \text{\fbox{$j_2$}} \cdots \text{\fbox{$j_n$}}$
        \State $s_i \gets s_0$
        \While{true}
            \State Desaloca $r$ tarefas de $s_i$
            \State $s \gets \text{o melhor planejamento possível baseado em $s_i$}$
            \If{o custo de $s$ é menor do que o de $s_i$}
                \State $s_i \gets s$
            \Else
                \State Break while loop
            \EndIf
        \EndWhile
        \State \Return $s_i$
    \end{algorithmic}
\end{algorithm}
