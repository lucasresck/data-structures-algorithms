\documentclass{article}

% Language setting
% Replace `english' with e.g. `spanish' to change the document language
\usepackage[portuguese]{babel}

% Set page size and margins
% Replace `letterpaper' with`a4paper' for UK/EU standard size
\usepackage[a4paper,top=2cm,bottom=2cm,left=3cm,right=3cm,marginparwidth=1.75cm]{geometry}

% Useful packages
\usepackage{amsmath}
\usepackage{graphicx}
\usepackage[colorlinks=true, allcolors=blue]{hyperref}
\usepackage{float}
\usepackage{algorithm}
\usepackage{algpseudocode}
\usepackage{float}

\graphicspath{{images/}}

\title{Projeto parcial \\
\large \textit{Closest string}}
\author{Lucas Emanuel Resck Domingues}

\begin{document}

    \maketitle
    
    \noindent \textbf{Tarefa 1} (1 ponto) Escolha um problema de otimização combinatória, apresente sua definição e apresente argumentos da sua NP-dificuldade (não é necessário fazer a redução ou a demonstração).

    \bigskip

    Vamos trabalhar o problema da \textbf{\textit{closest string}} (string mais próxima). Dadas $n$ strings de tamanho $m$, $s_1, \cdots, s_n$, buscamos uma nova string $s$ de tamanho $m$ tal que $\text{d}(s_i, s) \le d$, para todo $i \in \{1, \cdots, n\}$, porém com $d$ sendo o menor valor possível. A função $\text{d}(., .)$ é a distância de Hamming, \textit{i.e.}, o número de posições diferentes entre as duas strings. Vamos considerar, neste problema, um alfabeto fixo $\Sigma$.

Vejamos um exemplo. Suponhamos que queremos encontrar a string mais próxima a ambas as palavras GATO, GAME e GAMO. Então é óbvio que a solução será GA\_\_. Nossas opções para a terceira letra são T e M e para a quarta letra são O e E. Para cada combinação, temos o seguinte valor $d$ mínimo:
\begin{itemize}
    \item GATO $\Rightarrow 2$
    \item GATE $\Rightarrow 2$
    \item GAMO $\Rightarrow 1$
    \item GAME $\Rightarrow 2$
\end{itemize}
Ou seja, nossa solução é GAMO.

O problema da \textit{closest string} é NP-difícil.
De forma intuitiva, vamos pensar em um problema mais simples, em que o alfabeto é $\Sigma = \{0, 1\}$. Então a string assume a forma $001 \cdots 01$, que, por acaso, é a representação do problema da mochila 0-1. No problema da mochila, você é incentivado a colocar um item pois ele tem valor positivo, mas é desincentivado pois tem peso positivo. Você nunca sabe ``qual vale mais a pena'', se é deixar a mochila mais leve ou mais valiosa. Colocar um item, mesmo que leve e valioso, pode atrapalhar muito os outros itens. Isto é, não se sabe se é possível sacrificar um item para ter um ganho melhor com outros itens. O dilema é parecido nesse caso: você é incentivado a utilizar o caractere 0 pois isso vai se parecer com algumas \textit{strings}, porém é incentivado a utilizar o caractere 1 ao mesmo tempo, pois se parece com outras, e você não sabe qual vale mais a pena. Não se sabe se é possível sacrificar um caractere em uma posição para se ter um ganho melhor com outros caracteres em outras posições.

Claro, essa intuição não prova nada. Uma demonstração da NP-dificuldade do problema é dada por \cite{frances1997covering}, e não é nada trivial.

    \newpage
    
    \noindent \textbf{Tarefa 2} (2,5 pontos) Proponha um algoritmo exato para a resolução deste     problema. Calcule sua complexidade.

    \bigskip

    O problema é NP-difícil, então não precisamos ter medo de algoritmos com complexidade exponencial. Vamos propor aqui um algoritmo de força bruta, que testa todas as soluções possíveis que faz sentido serem testadas.

Sejam $s_1, \cdots, s_n$ as $n$ strings de tamanho $m$. Então, para cada posição $i$, a string $s$ pode assumir algum dos valores $S_i = \left\{s_1[i], \cdots, s_n[i]\right\}$. Nosso algoritmo será aquele que testa cada uma das combinações possíveis com $S_1$ para a posição 1 de $s$, $S_2$ para a posição 2 de $s$, $\cdots$, $S_m$ para a posição $m$ de $s$, e no final escolhe a melhor das soluções. Na verdade, é mais prático manter uma variável indicando a melhor das soluções durante a iteração e atualizando-a quando necessário.

Para uma determinada posição $i$, $S_i = \{s_1[i], \cdots, s_n[i]\}$ é todo o alfabeto $\Sigma$, no pior caso. Logo, como são $m$ posições, o número de combinações possíveis para $s$ é $O(|\Sigma|^m)$.

Para cada combinação de $s$, são necessários $n$ cálculos de distância de Hamming com as outras strings $s_i$, que, cada um, toma tempo $O(m)$.
Depois de calcular todos os $\text{d}(s, s_i)$, precisamos encontrar o maior desses valores, o que pode ser feito em tempo linear.
Ou seja, esse algoritmo tem complexidade $O((mn+n)|\Sigma|^m) = O(mn|\Sigma|^m)$.

    \newpage
    
    \noindent \textbf{Tarefa 3} (3 pontos) Proponha e implemente um algoritmo baseado em árvores
    ou por aproximação. Apresente argumentos para sua corretude e/ou aproximação e calcule sua complexidade.

    \bigskip

    \cite{li2002closest} propôs, e demonstrou, uma aproximação em tempo polinomial com razão de aproximação $1 + \epsilon$ para o problema da \textit{closest string}, como descrito na Tarefa 1.
A implementação do algoritmo não é trivial e não existe uma implementação desse algoritmo acessível, então não nos estenderemos nesse resultado. Buscarei descrever, e implementar, um algoritmo mais simples.


    \newpage

    \bibliographystyle{alpha}
    \bibliography{references}

\end{document}