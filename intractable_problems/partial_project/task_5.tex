Dada a busca local da Tarefa 4, é bastante conveniente aplicar aqui a meta-heurística GRASP. Não existe muito segredo: vamos construir uma heurística construtiva (aleatoriezada), aplicar a nossa busca local algumas vezes e repetir todo esse processo algumas boas vezes.

Iniciamos pela construção da heurística construtiva. Uma forma muito intuitiva é inicializar uma solução que possui, em cada posição, aquela letra que mais se repete na mesma posição das outras strings. Então, no nosso clássico exemplo, teríamos as strings
\fbox{G}\fbox{A}\fbox{T}\fbox{O}, \fbox{G}\fbox{A}\fbox{M}\fbox{E} e \fbox{G}\fbox{A}\fbox{M}\fbox{O},
que resultariam em uma solução inicial \fbox{G}\fbox{A}\fbox{M}\fbox{O}. Coincidentemente (ou não), essa é a solução ótima para esse exemplo.
Para introduzirmos a aleatoriedade, podemos restringir uma lista por tamanho, sorteando entre os caracteres que mais se repetem, controlando esse sorteio por meio de um parâmetro $\alpha$. O Algoritmo \ref{alg:grasp} descreve esse raciocínio com maiores detalhes.

\begin{algorithm}
    \caption{Meta-heurística GRASP para o problema da \textit{closest string}.}
    \label{alg:grasp}
    \begin{algorithmic}
        \Require Strings $s_1, \cdots, s_n$ de tamanho $m$
        \Require Número de iterações
        \Require Parâmetros $\alpha, r$
        \State Cria os conjuntos $S_i = \left\{s_1[i], \cdots, s_n[i]\right\}$, para cada $i$
        \State Ordena os elementos de $S_i$ pelo número de repetições
        \State Remove elementos repetidos de $S_i$
        \State Inicializa uma solução qualquer $s_0$
        \For{número de iterações}
            \State $s \gets \textsc{construcaoAleatoria()}$
            \State $s \gets \textsc{buscaLocal($s$)}$
            \State $s_0 \gets \textsc{atualizaSolucao($s$)}$
        \EndFor
        \State $d_0 \gets \text{custo de } s_0$
        \State \Return $s_0, d_0$
        \State
        \Procedure{construcaoAleatoria}{}
            \State $s \gets \text{empty string}$
            \For{cada posição $1 \le i \le m$}
                \State $s_i \gets$ sorteia dentre os $\alpha|S_i|$ candidatos que mais se repetem
                \State $s \gets s + s_i$
            \EndFor
            \State \Return $s$
        \EndProcedure
        \State
        \Procedure{buscaLocal}{$s$}
            \State Apaga aleatoriamente $r$ caracteres de $s$
            \State $s \gets$ busca pela solução que melhor preenche $s$ \Comment{Possivelmente uma recursão.}
            \State \Return $s$
        \EndProcedure
    \end{algorithmic}
\end{algorithm}

Construir os conjuntos $S_i$ de candidatos tem complexidade $O(nm\ln m)$, por causa do ordenamento. A construção aleatória toma tempo linear em $m$, pois cada sorteio é constante. A busca local testa todas as possibilidades para os $r$ elementos faltantes. No pior caso, são $|\Sigma|^r$ possibilidades, onde cada uma toma tempo $O(mn)$ para o cálculo do custo $d$. Atualizar a solução envolve calcular o custo em tempo $O(mn)$. Finalmente, se $N = \text{número de iterações}$, então temos como complexidade para nosso algoritmo
\[O(nm\ln m + N(m + mn|\Sigma|^r + mn)) = O(nm\ln m + Nmn|\Sigma|^r).\]
Assumimos, a princípio, que $\Sigma$ é fixo; $N, r$ são parâmetros fixos também. Logo, a complexidade final do algoritmo é $O(nm\ln m)$.

Vejamos um exemplo de aplicação da nossa meta-heurística. Suponhamos que possuímos as seguintes sequências de DNA:
\begin{itemize}
    \item \fbox{A}\fbox{T}\fbox{C}\fbox{T}\fbox{A}\fbox{\color{blue}{T}}\fbox{A}\fbox{G}\fbox{\color{blue}{A}}\fbox{A}\fbox{\color{blue}{G}}\fbox{\color{blue}{T}}
    \item \fbox{A}\fbox{T}\fbox{C}\fbox{T}\fbox{A}\fbox{\color{blue}{C}}\fbox{A}\fbox{G}\fbox{\color{blue}{T}}\fbox{A}\fbox{\color{blue}{A}}\fbox{\color{blue}{C}}
    \item \fbox{A}\fbox{T}\fbox{C}\fbox{T}\fbox{A}\fbox{\color{blue}{C}}\fbox{A}\fbox{G}\fbox{\color{blue}{A}}\fbox{A}\fbox{\color{blue}{G}}\fbox{\color{blue}{T}}
    \item \fbox{A}\fbox{T}\fbox{C}\fbox{T}\fbox{A}\fbox{\color{blue}{T}}\fbox{A}\fbox{G}\fbox{\color{blue}{A}}\fbox{A}\fbox{\color{blue}{G}}\fbox{\color{blue}{T}}
\end{itemize}
As posições em azul correspondem àquelas que contém letras diferentes entre as strings. Aplicando nossa meta-heurística GRASP, conseguimos o resultado \fbox{A}\fbox{T}\fbox{C}\fbox{T}\fbox{A}\fbox{\color{blue}{C}}\fbox{A}\fbox{G}\fbox{\color{blue}{A}}\fbox{A}\fbox{\color{blue}{G}}\fbox{\color{blue}{C}}, com custo de $d=2$. 
