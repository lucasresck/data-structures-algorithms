Vamos trabalhar o problema da \textbf{\textit{closest string}} (``string mais próxima'', em tradução livre). Dadas $n$ strings de tamanho $m$, $s_1, \cdots, s_n$, buscamos uma nova string $s$ de tamanho $m$ tal que $\text{d}(s_i, s) \le d$, para todo $i \in \{1, \cdots, n\}$, porém com $d$ sendo o menor valor possível. Em geral, chamaremos $d$ de custo, pois é o valor que queremos minimizar. A função $\text{d}(\cdot, \cdot)$ é a distância de Hamming, \textit{i.e.}, o número de posições diferentes entre duas strings. Vamos considerar, neste problema, um alfabeto fixo $\Sigma$.
De forma um pouco mais abstrata, estamos buscando no espaço métrico de hamming $\left(\Sigma^m, \text{d}\right)$ a menor bola fechada, com centro $s$ e raio $d$, que contém todas as $n$ strings $s_1, \cdots, s_n$.

Vejamos um exemplo. Suponhamos que queremos encontrar a string mais próxima a ambas as strings \fbox{G}\fbox{A}\fbox{T}\fbox{O}, \fbox{G}\fbox{A}\fbox{M}\fbox{E} e \fbox{G}\fbox{A}\fbox{M}\fbox{O}. Então é óbvio que a solução será \fbox{G}\fbox{A}\fbox{?}\fbox{?}. Nossas opções para a terceira letra são T e M e para a quarta letra são O e E. Para cada combinação, temos o seguinte valor $d$ mínimo:
\begin{itemize}
    \item \fbox{G}\fbox{A}\fbox{T}\fbox{O} $\Rightarrow 2$
    \item \fbox{G}\fbox{A}\fbox{T}\fbox{E} $\Rightarrow 2$
    \item \fbox{G}\fbox{A}\fbox{M}\fbox{O} $\Rightarrow 1$
    \item \fbox{G}\fbox{A}\fbox{M}\fbox{E} $\Rightarrow 2$
\end{itemize}
Ou seja, nossa solução é \fbox{G}\fbox{A}\fbox{M}\fbox{O}.

O problema da \textit{closest string} é NP-difícil.
De forma intuitiva, vamos pensar em um problema mais simples, em que o alfabeto é $\Sigma = \{0, 1\}$. Então a string assume a forma $\textsc{\fbox{0}}\textsc{\fbox{0}}\textsc{\fbox{1}} \cdots \textsc{\fbox{0}}\textsc{\fbox{1}}$, por exemplo, que por acaso é a representação do problema da mochila 0-1. No problema da mochila, você é incentivado a colocar um item pois ele tem valor positivo, mas é desincentivado pois tem peso positivo e há restrição de peso. Você nunca sabe ``qual vale mais a pena'', se é deixar a mochila mais leve ou mais valiosa. Colocar um item, mesmo que leve e valioso, pode atrapalhar muito os outros itens, a ponto de deixar a mochila menos valiosa ao final. Isto é, não se sabe se é possível sacrificar um item valioso para ter um ganho melhor com outros itens, possivelmente menos valiosos. O dilema é parecido nesse caso: você é incentivado a utilizar o caractere 0 pois isso vai se parecer com algumas \textit{strings}, porém é incentivado a utilizar o caractere 1 ao mesmo tempo, pois se parece com outras, e você não sabe qual vale mais a pena. Não se sabe se é possível sacrificar um ``bom'' caractere (aquele que diminuiria um erro local, por exemplo) em uma posição para se ter um ganho melhor com outros caracteres em outras posições, e vice-versa.

Claro, essa intuição não prova nada. Uma demonstração da NP-dificuldade do problema é dada por \citeauthor{frances1997covering} \cite{frances1997covering}. A demonstração não é trivial e, acredito, não cabe neste trabalho.