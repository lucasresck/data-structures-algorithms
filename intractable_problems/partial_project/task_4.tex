Suponhamos que temos a solução
\[s = \text{\fbox{$s_1$}\fbox{$s_2$}$\cdots$\fbox{$s_{m-1}$}\fbox{$s_m$}}.\]
Uma busca local bem simples, porém intuitiva e óbvia, é apagar aleatoriamente $r \le m$ letras e preencher a solução da melhor forma possível. São, ao todo, $|\Sigma|^r$ possibilidades, no pior caso, sendo que para checar cada uma toma-se tempo $O(mn+n) = O(mn)$, como descrito na Tarefa 2: $n$ cálculos da distância de Hamming, que tomam $O(m)$ cada uma, e o cálculo de um máximo ao final, que é feito em tempo linear $O(n)$. Logo, a complexidade dessa busca local é $O(mn|\Sigma|^r) = O(mn)$, dado que $\Sigma$ é fixo por hipótese e $r$ é um parâmetro fixo.
A implementação dessa busca local será feita na Tarefa 5, Algoritmo \ref{alg:grasp}, pois será utilizada na meta-heurística.
